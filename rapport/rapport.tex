
\documentclass{article}
\usepackage{geometry} % see geometry.pdf on how to lay out the page. There's lots.
\geometry{a4paper} % or letter or a5paper or ... etc
\usepackage[T1]{fontenc}
\usepackage[utf8]{inputenc}
\usepackage[french,english]{babel}

% \geometry{landscape} % rotated page geometry

% See the ``Article customise'' template for come common customisations

\title{Rapport intermédaire du projet long \\ La cabane à oiseaux connectée }
\author{Harmony Simon-Duchatel, Clément Blérald}

%%% BEGIN DOCUMENT
\begin{document}

\maketitle


\section{Présentation}
\paragraph {}
Une cabanes à oiseaux est un bon moins d'attirées mésanges et moineaux. Mais il est fort probable que d'autres espèces moins appréciés viennent s'y nourrir. \\
C'est pour cela que nous proposons de créé une cabane à oiseaux connectées, capable de détecter une présence dans la cabane et de déterminer l'espace de l'oiseau s'y trouvant. 

\section{Réalisation actuel}
\paragraph {}
Pour débuter, nous avons établis un fichier déterminant les principaux oiseaux que nous serrons capables d'attirée à Paris, avec leurs caractéristiques. \\
Par la suite 



\end{document}